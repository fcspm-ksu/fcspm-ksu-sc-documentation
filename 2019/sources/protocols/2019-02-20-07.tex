\documentclass[
	a4paper,
	12pt,
	oneside,
	%draft
]{extreport}

% Кодировка, шрифты и языки
\usepackage{cmap}
\usepackage[T2A]{fontenc}
\usepackage[utf8]{inputenc}
\usepackage[english,russian,ukrainian]{babel}

% Лист и структура
\usepackage[a4paper, top=1cm, bottom=1cm, left=3cm, right=1cm]{geometry}
\usepackage{indentfirst}
\usepackage{enumitem}
\usepackage{multicol}

% Разное
\usepackage{datetime}
\usepackage{ifthen}
\usepackage{lipsum}
\usepackage{url}

% Абзацный отступ
\parindent=1.25cm
% ----------------------------------------------------------------------------
% ---------------------------HEADING------------------------------------------
\newcommand{\heading}{\begin{center}
\parindent=0cm\parskip=-0.1cm\bfseries\footnotesize
\par МІНІСТЕРСТВО ОСВІТИ І НАУКИ УКРАЇНИ
\par ХЕРСОНСЬКИЙ ДЕРЖАВНИЙ УНІВЕРСИТЕТ
\par ФАКУЛЬТЕТ КОМП'ЮТЕРНИХ НАУК, ФІЗИКИ ТА МАТЕМАТИКИ
\parskip=-0.2cm\par\large СТУДЕНТСЬКА РАДА ФАКУЛЬТЕТУ
\par\hrule width \hsize height 1mm \kern 0.25mm \hrule width \hsize height 0.25mm
\end{center}}
% ----------------------------------------------------------------
% ---------------------------REQUISITES------------------------------------------
\newcommand{\Requisites}[4]{\par\noindent
\begin{minipage}[t]{80mm}
	\begin{minipage}[t]{\textwidth}
		{\par\noindent\textbf{№#1 від #2}}
	\end{minipage}
	\par	
	\begin{minipage}[t]{\textwidth}
		{\par\noindent #3}	
	\end{minipage}
\end{minipage}
\hfill
\begin{minipage}[t]{60mm}
	\par\noindent #4
\end{minipage}
}
% ---------------------------------------------------------------------
% ---------------------------PROTOCOL----------------------------------
\newcommand{\Attendees}[1]{{\par\noindent ПРИСУТНІ: \par #1}}	
\newcommand{\Invented}[1]{{\par\noindent ЗАПРОШЕНІ: \par #1}}	

\newcommand{\topic}[1]{\item #1}
\newenvironment{protocolAgenda}
	{\bigskip\par\noindent ПОРЯДОК ДЕННИЙ:
		\begin{enumerate}[topsep=0pt,itemsep=-1ex,partopsep=1ex,parsep=1ex]
}
	{\end{enumerate}}
	
\newcommand{\protocolVoting}[3]{
	\par Результати голосування: <<за>> -- #1; <<проти>> -- #2; <<утримались>> -- #3.}

\newcounter{ProtocolItemizator}
\setcounter{ProtocolItemizator}{1}

\newcommand{\protocolItem}[3]{
	\bigskip
	\par\noindent \Roman{ProtocolItemizator}. СЛУХАЛИ:\par #1 
	\par\noindent ВИСТУПИЛИ: \par #2
	\par\noindent УХВАЛИЛИ:\par #3
	\stepcounter{ProtocolItemizator}}
% ---------------------------------------------------------------------
% ---------------------------SIGNS----------------------------------
\newcommand{\sign}[2]{
	\bigskip\par\noindent
	\begin{minipage}[b]{70mm}#1\end{minipage}	\hfill
	\begin{minipage}[b]{50mm}#2\end{minipage}	\par	}
	
\newcommand{\signEqualProof}[3]{
	\bigskip\vfill
	\par\noindent #1
	\par\noindent Згідно з оригіналом
	\sign{#2}{#3}\vfill}
% ---------------------------------------------------------------------
% ---------------------------------------------------------------------
\begin{document}
\pagestyle{empty}
\heading
\Requisites
{07}
{20 лютого 2019 р.}
{}{\hfill\bfseries  м. Херсон}

\begin{center}\textbf{ПРОТОКОЛ}\end{center}

\par\noindent Головуючий засідання: Головачова А.І.
\par\noindent Секретар: Сенчишен Д.О.

\Attendees{
%Бродовський~А.А.,
Бельза~К.М.,
Височенко~Г.А.,
Вовчанчина~Т.І.,
Воропаєва~І.В.,
%Головачова~А.В.,
%Іванова~О.В.,
Карпенко~К.В.,
Лукавий~С.В.,
Магурян~Н.Д.,
%Лапік~С.В.,
%Овчаренко~В.О.,
Тарасюк~А.О.
}

\Invented{декан факультету Кушнір Н.О.}

\begin{protocolAgenda}
\topic{Про склад студентської ради факультету.}
\topic{Про роботу студентського самоврядування факультету.}
\end{protocolAgenda}

\protocolItem
{Головачову А.І. про склад студентської ради факультету.
}
{
Головачова А.І. представила присутнім нових учасників студентського самоврядування факультету та напрями їх діяльності, зокрема:

\begin{enumerate}[topsep=0pt,itemsep=-1ex,partopsep=0ex,parsep=1ex, label={\arabic*)}]
\item Тарасюка А. як заступника голови ради;
\item Карпенко К. як голову комітету здоров'я (окремо або у складі спортивного сектору);
\item Бельзу К. як голову сектору масс-медіа та внутрішніх комунікацій;
\item Височенко Г. як відповідальну за культурно-масову організацію;
\item Лукавого С. та Магуряна Н. як учасників трудового сектору.
\end{enumerate}

}
{
Інформацію прийняти до відома. Затвердити склад після обрання голови.
}

\protocolItem
{Головачову А.І. роботу студентського самоврядування факультету.
}
{
Кушнір Н.О., Головачова А.І. представили присутнім найближчі плани та події, що потребують підготовки, а такою озвучили елементи стратегії роботи самоврядування факультету, зокрема:

\begin{enumerate}[topsep=0pt,itemsep=-1ex,partopsep=0ex,parsep=1ex, label={\arabic*)}]
\item свято до Міжнародного дня жінок;
\item фінал турніру <<Що? Де? Коли?>>, орієнтовно 26-28 лютого;
\item фестиваль <<Молода хвиля>>, орієнтовно середина квітня, в т.ч. - кількість людей для виконання конкретних задач (звук, світло, мікрофони, костюми тощо);
\item збір макулатури та банка для батарейок;
\item помістити стенди на стіни в коридорах;
\item публікація новин у всіх соц. мережах та на сайті університету;
\item звітність про роботу студентського самоврядування (в т.ч. - фінансова);
\item просування факультету та університету взагалі в рейтингах та соц. мережах;
\item рисунки на стіні поряд з лабораторією робототехніки;
\item шпалери для малювання крейдою;
\item футболки з символікою факультету.
\end{enumerate}
}
{
Інформацію та звіт Кушнір Н.О. про роботу факультету прийняти до відома.
}

\bfseries 
\sign{Головуючий засідання}{А.І.~Головачова}
\sign{Секретар}{Д.О.~Сенчишен}
\normalfont

\end{document}
