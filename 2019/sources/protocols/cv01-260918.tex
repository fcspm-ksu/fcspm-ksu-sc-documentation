\documentclass[
	a4paper,
	12pt,
	oneside,
	draft
]{extreport}

% Кодировка, шрифты и языки
\usepackage{cmap}
\usepackage[T2A]{fontenc}
\usepackage[utf8]{inputenc}
\usepackage[english,russian,ukrainian]{babel}

% Лист и структура
\usepackage[a4paper, top=1cm, bottom=1cm, left=3cm, right=1cm]{geometry}
\usepackage{indentfirst}
\usepackage{enumitem}
\usepackage{multicol}

% Разное
\usepackage{datetime}
\usepackage{ifthen}
\usepackage{lipsum}
%%%
%\usepackage[printwatermark]{xwatermark}
%\usepackage{xcolor}
%\newwatermark[allpages,color=red!50,angle=45,scale=3,xpos=0,ypos=0]{ПРОЕКТ}
%%%
% Абзацный отступ
\parindent=1.25cm
% ----------------------------------------------------------------------------
% ---------------------------HEADING------------------------------------------
\newcommand{\heading}{\begin{center}
\parindent=0cm\parskip=-0.1cm\bfseries\footnotesize
\par МІНІСТЕРСТВО ОСВІТИ І НАУКИ УКРАЇНИ
\par ХЕРСОНСЬКИЙ ДЕРЖАВНИЙ УНІВЕРСИТЕТ
\par ФАКУЛЬТЕТ КОМП'ЮТЕРНИХ НАУК, ФІЗИКИ ТА МАТЕМАТИКИ
\parskip=-0.2cm\par\large КОНФЕРЕНЦІЯ СТУДЕНТІВ ФАКУЛЬТЕТУ
\par\hrule width \hsize height 1mm \kern 0.25mm \hrule width \hsize height 0.25mm
\end{center}}
% ----------------------------------------------------------------
% ---------------------------REQUISITES------------------------------------------
\newcommand{\Requisites}[4]{\par\noindent
\begin{minipage}[t]{80mm}
	\begin{minipage}[t]{\textwidth}
		{\par\noindent\textbf{№#1 від #2}}
	\end{minipage}
	\par	
	\begin{minipage}[t]{\textwidth}
		{\par\noindent #3}	
	\end{minipage}
\end{minipage}
\hfill
\begin{minipage}[t]{60mm}
	\par\noindent #4
\end{minipage}
}
% ---------------------------------------------------------------------
% ---------------------------PROTOCOL----------------------------------
\newcommand{\Attendees}[1]{{\par\noindent ПРИСУТНІ: \par #1}}	
\newcommand{\Invented}[1]{{\par\noindent ЗАПРОШЕНІ: \par #1}}	

\newcommand{\topic}[1]{\item #1}
\newenvironment{protocolAgenda}
	{\bigskip\par\noindent ПОРЯДОК ДЕННИЙ:
		\begin{enumerate}[topsep=0pt,itemsep=-1ex,partopsep=1ex,parsep=1ex]}
	{\end{enumerate}}
	
\newcommand{\protocolVoting}[3]{
	\par Результати голосування: <<за>> -- #1; <<проти>> -- #2; <<утримались>> -- #3.}

\newcounter{ProtocolItemizator}
\setcounter{ProtocolItemizator}{1}

\newcommand{\protocolItem}[3]{
	\bigskip
	\par\noindent \Roman{ProtocolItemizator}. СЛУХАЛИ:\par #1 
	\ifthenelse{\equal{#2}{}}{}{\noindent
	\par\noindent ВИСТУПИЛИ: \par #2}
	\par\noindent УХВАЛИЛИ:\par #3
	\stepcounter{ProtocolItemizator}}
% ---------------------------------------------------------------------
% ---------------------------SIGNS----------------------------------
\newcommand{\sign}[2]{
	\bigskip\par\noindent
	\begin{minipage}[b]{70mm}#1\end{minipage}	\hfill
	\begin{minipage}[b]{50mm}#2\end{minipage}	\par	}
	
\newcommand{\signEqualProof}[3]{
	\bigskip\vfill
	\par\noindent #1
	\par\noindent Згідно з оригіналом
	\sign{#2}{#3}\vfill}
	
\newcommand{\writerTel}[2]{
	\vfill\scriptsize\parindent=-0.5mm
	\par\noindent #1
	\par\noindent #2}
% ---------------------------------------------------------------------
% ---------------------------FOR REQUEST EXTRACT------------------------------
\newcommand{\groupElement}[6]{\vbox{
\par\noindent\hrulefill
\par\noindent {
\scriptsize На {\footnotesize \textbf{#4}} \hfill з заяви №#1 від #2  \hfill <<#3>> \hfill \textit{Згоден {\tiny /підпис/} Кузьмич~В.І.}}
\par\noindent \textbf{#5} група: #6}}
% ----------------------------------------------------------------------------
% ----------------------------------------------------------------------------
% ----------------------------------------------------------------------------
%\usepackage{datetime}
%Версія від \today\ \currenttime
% ---------------------------------------------------------------------

\begin{document}
\thispagestyle{empty}
\heading
\Requisites
{1}
{26 вересня 2018 р.}
{}{\hfill\bfseries м. Херсон}

\begin{center}\textbf{ВИТЯГ З ПРОТОКОЛУ}\end{center}

\par\noindent Голова: Чебан В.А.
\par\noindent Секретар: Сенчишен Д.О.

\Attendees{
студенти факультету комп'ютерних наук, фізики та математики.}

\begin{protocolAgenda}
\topic{Про обрання делегатів на загальноуніверситетську конференцію студентів ХДУ.}
\end{protocolAgenda}

\protocolItem
{Чебан В.А. про порядок обрання делегатів з числа студентів, які мають бути обрані конференцією студентів факультету на загальноуніверситетську конференцію студентів ХДУ.}
{}
{За простою більшістю отриманих голосів вважати обраними делегатами на загальноуніверситетську конференцію студентів ХДУ 27 вересня 2018 р.:

\bigskip
\noindent\begin{minipage}{0.7\textwidth}
\begin{enumerate}[topsep=0pt,itemsep=-1ex,partopsep=1ex,parsep=1ex]
\item Овчаренко Вікторія Олегівна \hfill --- група 15-221
\item Тарасюк Артур Олександрович \hfill --- група 15-231
\item Варава Олександр Олександрович \hfill --- група 15-431
\item Воробйов Євген Андрійович \hfill --- група 15-431
\item Давидок Олександр Юрійович \hfill --- група 15-431
\item Іванова Олена Володимирівна \hfill --- група 15-431
\item Нагачевський Андрій Олександрович \hfill --- група 15-431
\item Пулінець Анастасія Юріївна \hfill --- група 15-431
\item Сенчишен Денис Олександрович \hfill --- група 15-431
\item Чебан Вікторія Андріївна \hfill --- група 15-431
\item Шапарьов Олексій Юрійович \hfill --- група 15-431

\end{enumerate}
\end{minipage}
}



\bfseries 
\sign{Голова {\hfill\normalfont\small\itshape /підпис/} }{В.А. Чебан}
\sign{Секретар {\hfill\normalfont\small\itshape /підпис/} }{Д.О.~Сенчишен}
\normalfont


\signEqualProof{26 вересня 2018 р.}{Секретар студентської ради \\ факультету комп'ютерних наук, \\ фізики та математики ХДУ}{Д.О. Сенчишен}


\end{document}

